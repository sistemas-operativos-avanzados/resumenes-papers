Un monitor de máquina virtual (\emph{virtual machine monitor} VMM) es un sistema de software que particiona un máquina física en múltiples máquinas virtuales. Tradicionalmente, los VMMs crean replicas precisas de la máquina subyacente a través de una emulación fiel. Los VMMs soportan la ejecución de sistemas operativos invitados(\emph{guest}) legados como Windows o Linux sin necesidad de modificaciones. Un gropo de investigación de la Universidad de Washington ha desarrollado Denali VM, con la premisa qeu es posible y útil el considerar una abstracción de VM que difiera de una máquina física. Los dos mayores resultados de este esfuerzo son paravirtualización e interimposición de hardware. 

\paragraph{\textnormal{\textbf{VMM Advantages:}}}
Recientemente, los VMMs han experimentado un renacimiento debido en gran parte al éxito de la monitor de máquina virtual de VMware. Algunos factores contribuyen a la popularidad actulas de los VMMs: \underline{\emph{Implementación simple:}} en comparación con un SO completo como Linux o Windows. Los VMMs logran simplicidad al evitar la implementación de abstracciones de alto nivel como \emph{sockets} TCP/IP y sistemas de archivos. Esta simplicidad hace que los VMMs sean muy adecuados para aspectos de fiabilidad y seguridad del sistema. Además, su simplificidad los hace más fácil de extender y modificar que los SO tradicionales. \underline{\emph{Whole-system services:}} Una máquina virtual(VM) captura un sistema completo de software, incluyendo el SO y su grupo de aplicaciones. Esto es importante porque muchos servicios de interés transcienden mecanismos de encapsulación tradicionales como espacios de direccioens o procesos de SO. \underline{\emph{Soporte para sistemas operativos invitados legados:}} La habilidad de correr múltiples SO legados en una sola máquina ha probado ser muy útil con el tiempo. Hoy en día los administradores de sistemas usan esta capacidad para consolidar varios servidores poco utilizados en una sola máquina y para forzar aislamiento para código inseguro o no confiable. \underline{\emph{Rendimiento tolerable:}} Historicamente los VMMs han sufrido del inconveniente de un rendmiento lento en relacion con arquitecturas convencionales de sistemas. Conforme la velocidad de procesador ha aumentado, esta penalidad por virtualización se ha vuelto tolerable en muchas configracioens. Además, recientes avances en el diseño de VMMs han llevado a que el costo de virtualización sea menor. 

\paragraph{\textnormal{\textbf{Scaling a VMM:}}}  
La escalabilidad se refiere a la habilidad de correr muchas VMs en un sola máquina física. Dos observaciones motivaron el interés en la escalabilidad: (1) Han emergido dominios de aplicación que requieren mínimo o esporádico tiempo de procesador. (2) La ley de Moore ha producido una abundancia de poder crudo de CPU, habilitando colocación de muchos servicios para disminuir la sobrecarga administrativa. La investigación ha revelado que los VMMs tradicionales sufren de cuellos de bottella en escalabilidad los cuales artificialmente reestringuen el número de VMs que un sistema puede soportar. Estos cuellos de botella existen porque la noción de tiempo es más compleja en un VMM. Un VMM corre múltiples VMs en paralelo, asi que cada VM solo corre en el procesador de la máquina real por $1/N$ del total del tiempo de CPU en promedio. Este efecto crea una noción de \emph{tiempo virtual} que avanza a una proporción diferente que el reloj físico. Conforme el número de VM incremeta, la separación entre tiempo virtual y tiempo físico aumenta, afectando adversamente cualquier aspecto de hardware dependiente de tiempo, incluyendo entrega de interrupciones y temporizadores. Para abordar estos desafíos, se propone paravirtualización. La idea clave de esta técnica es el exponer una arquitectura virtual de hardware que difiere de la arquitectura del hardware físico subyacente. Pequeños cambios a la arquitectura virtual son suficientes para eliminar los cuellos de botella artificiales que afectuan los sistemas tradicionales.  


\section{¿Cuál es el problema que plantea el \textit{paper}?}

\section{¿Por qué el problema es interesante o importante?}

\section{¿Qué otras soluciones se han intentado para resolver este problema?}
     
\section{¿Cuál es la solución propuesta por los autores?}

\section{¿Qué tan exitosa es esta solución?} 