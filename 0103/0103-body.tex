El concepto de interrupción se ha expandido a través de los años. Diferentes fabricantes usan términos como \textit{exceptions, faults, aborts \textnormal{y} interrups} para describir este comportamiento. No hay un claro concenso ni un significado exacto.

\paragraph{\textnormal{\textbf{80x86 Interrup Structure and Interrup Services Routines(ISRs)}}}

\paragraph{\textnormal{\textbf{Traps}}}

\paragraph{\textnormal{\textbf{Exceptions}}}

\begin{enumerate}
    \item \textit{Divide Error Exception \texttt{(INT 0)}}
    \item \textit{Single Step (Trace) Exception \texttt{(INT 1)}}
    \item \textit{Breakpoint Exception \texttt{(INT 3)}}
    \item \textit{Overflow Exception \texttt{(INT 4/INTO)}}
    \item \textit{Bounds Exception \texttt{(INT 5/BOUND)}}
    \item \textit{Invalid Code Exception \texttt{(INT 6)}}
    \item \textit{Coprocessor Not Available \texttt(INT 7)}
\end{enumerate}

\paragraph{\textnormal{\textbf{Hardware Interrupts}}}

\paragraph{The 8559A Programable Interrup Controller (PIC)}

\begin{enumerate}
    \item \textit{The Timer Interrup \texttt{(INT 8)}}
    \item \textit{The Keyboard Interrup \texttt{(INT 9)}}
    \item \textit{The Serial Port Interrups \texttt{(INT 0Bh} and \texttt{INT 0Ch)}}
    \item \textit{The Parallel Port Interrupts \texttt{(INT 0Dh} and \texttt{INT 0Fh)}}
    \item \textit{The Diskett and Hard Drive Interrups \texttt{(INT 0Eh} and \texttt{INT 76h}}
    \item \textit{The Real-Time Clock Interrup \texttt{(INT 70h)}}
    \item \textit{The FPU Interrup \texttt(INT 75h)}
    \item \textit{Nonmaskable Interrups \texttt{(INT 2)}}
    \item \textit{Other Interrups}
\end{enumerate}

\paragraph{\textnormal{\textbf{Chaining Interrup Service Routines}}}

\paragraph{\textnormal{\textbf{Reentrancy Problems}}}

\paragraph{\textnormal{\textbf{The Efficiency of an Interrup Driven System}}}

\paragraph{Interrup Driven I/O vs. Polling}

\paragraph{Interrup Service Time}

\paragraph{Interrup Latency}

\paragraph{Prioritized Interrups}

\paragraph{Debuggin ISRs}

\section{¿Cuál es el problema que plantea el \textit{paper}?}

\section{¿Por qué el problema es interesante o importante?}

\section{¿Qué otras soluciones se han intentado para resolver este problema?}     

\section{¿Cuál es la solución propuesta por los autores?}

\section{¿Qué tan exitosa es esta solución?}
